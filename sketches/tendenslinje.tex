\documentclass[a4paper,11pt]{article}

\usepackage{revy}
\usepackage[utf8]{inputenc}
\usepackage[T1]{fontenc}
\usepackage[danish]{babel}

\revyname{Biorevy}
\revyyear{2017}
% HUSK AT OPDATERE VERSIONSNUMMER
\version{0.1}
\eta{5.27 min}
\status{Færdig}
\responsible{tex@biorevy.dk}

\title{Tendenslinje}
\author{Stine \& BT}

\begin{document}
\maketitle

\begin{roles}
    \role{MilPe}[Karla] Milena Penkowa
    \role{Djævl}[Ditte] Djævel
    \role{Ninja}[Isabella] Ninja
    \role{Ninja}[Henriette] Mikrofon-ninja
\end{roles}

\begin{props}
    \prop{Error bars}[Person, der skaffer] Bongruller eller lignende
    \prop{konklusion} [Person, der Skriver]
\end{props}


\begin{sketch}

\scene{Beskrivelse}
\scene{Sort lys ned}

\scene En tendenslinje står på scenen, og chiller. En studerende kommer ind med en kasse, hvorpå der står ``DATA''. Ninjaen åbner kassen, kigger lidt i papirerne. Tager papirer og random ting ud af kassen. Tænker sig lidt om, og retter så på tendenslinien. 
\scene S står i mellemtiden og ringer til mor. 

\says{Stud} Hej mor! Ja, det går rigtig godt, jeg er lige blevet færdig i lab. Ja, så mangler jeg kun databehandlingen. 
Nej nej, der er en hel uge til jeg skal aflevere PhD'en! Ej, Hvad mener du?! Metodeafsnittet er jo bare sådan noget form-noget. Ej, det kan da ikke tage så lang tid. 
Nej, jeg har helt styr på det, har allerede skrevet konklusionen.  
\scene{Ninja giver Stud konklusion .Tend, Stiller sig op som et S, med hjælp fra ninjaen} 
\says{Stud} Okay. Yes, vi har noget data. 
\scene S går lidt rundt om tendenslinjen, ser tænksomt på den. 
\says{Stud} Men\ldots det passer jo ikke så godt med min konklusion. Det må vi lige få gjort noget ved. Kan vi ikke lige tage de her punkter fra? Lad os sige det er outliers\ldots
\says{Ninja}\ldots jo, okay. 
\scene Gør tendensen mere lige.
\says{Stud} Det er vist fortegn. (sagt til dvævel) Det ser meeeget bedre ud! Lad os lige se på det. Hmm lad os lige få nogle error bars på
\scene N lader en bonrulle falde fra bag ved tendensliniens arm, som trilller hen over scenen 
\says{Stud} øh, nej, lad os lade være med at have error bars på. \act {begynder at vikle bonrullen rundt om armen}\says Lad os glemme error bars, og i stedet se fremad! Hvordan ser modellen så ud i 2020? 
\scene Ninjaen hiver i tendenslinien
\says{Ninja} \ldots jeg kan altså ikke nå. 
\says{Stud} okay, men så tror jeg \act{trækker en linje i luften og estimerer hvor højt det er} \says ca så meget. Det passer fint med min konklusion. 
\says{Stud} Dygtig model \act{klapper tendenslinjen på hovedetKigger på sin disposition} Nå, næste del af databehandlingen. Lad os så se på sammenhængen mellem mine parametre. 
\act{T breakdancer}Truende Hvis du ikke opfører dig ordentligt kommer jeg med det dobbeltlogaritmiske papir! 
\says{Tend} Kan du slet ikke lide mine kurver? 
\says{Stud} NEJ, du er mega besværlig. 
\says{Ninja} Hvis du nu havde lavet noget bedre data, så kunne det være jeg kunne gøre mit job ordentligt.  
\says{Stud} \act{hører ikke efter} Jeg skal altså bruge en ordentlig graf til min opgave. Hvad nu hvis vi sorterer musenes alder? 
\scene T bliver helt skævvreden
\says{Stud} hmm, hvad så hvis vi sorterer efter hjernernes størrelse?
\scene Ninjaen roder alle papirer igennem, frustreret. 
\says{Ninja}Du har jo for helvede ikke målt størrelsen af noget\ldots
\says{Stud} Hm\ldots okay okay, tallene har jo en størrelse, så hvad hvis vi sorterer efter TALLENES størrelse?
\says{Tend}\ldots suk. \act{Bliver helt lineær}. Det her er RIGTIG dårlig videnskab. 
\says{Stud} Se DÉT er en flot kurve!

\scene{Sort lys op.}

\end{sketch}
\end{document}
