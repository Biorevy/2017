\documentclass[a4paper,11pt]{article}

\usepackage{revy}
\usepackage[utf8]{inputenc}
\usepackage[T1]{fontenc}
\usepackage[danish]{babel}
\usepackage{hyperref}

\revyname{Biorevy}
\revyyear{2017}
% HUSK AT OPDATERE VERSIONSNUMMER
\version{0.1}
\eta{7.23 min}
\status{Færdig}
\responsible{tex@biorevy.dk}

\title{Planula i krise}
\author{Tristan og Jonathan}

\begin{document}
\maketitle

\begin{roles}
    \role{Planu}[Aske] Planula 
    \role{Motor}[Tristan] Motorneuron 
    \role{Alge}[Jeppe] Alge - efterfølgende InvisiChoir
    \role{Cot}[Emil] Crown of Thorns-søpindsvin 
    \role{Kor}[Camilla] InvisiChoir
    \role{Kor}[Caroline] InvisiChoir
    \role{Kor}[Karla] InvisiChoir
    \role{Kor}[Kristina] InvisiChoir
    \role{Ninja}[Henriette]
    \role{Ninja}[Morten Finne]
\end{roles}

\begin{props}
    \prop{1 fatboy}[skaffer selv]
    \prop{kontrollers}[skaffer selv]
    \prop{pizza bakker} [skaffer selv]
\end{props}

\textbf{Kommentarer/kontekst fra forfatterne:}
En planula bliver født, måske med noget god gammeldags koralporno på AV
\url{https://www.youtube.com/watch?v=06BPlLATtgc}.

Den møder sit motor-neuron (enten som en rekvisit med speak eller en reel karakter) og bliver virkelig glad for at den kan bevæge sig. På et et tidspunkt finder den ud af alle de forfærdelige ting der sker i verden (der er terror, nedskæringer, rydning af Amager Fælled, KU’s rektor hedder ikke Ralf osv.).
Derfor vil den ikke bevæge sig mere, motor neuronet siger den skal ellers kan den jo ikke finde næring. På det tidspunkt (lige før vores planula opgiver) kommer en alge forbi og siger den godt kan hjælpe, hvis den bare får et sted at bo og noget “start-næring” planulaen tænker over hvad den skal give den, men finder hurtigt frem til at det da er dens motor neuroner. Algen og planulaen (der nu er blevet en koral) sidder og samarbejder om projekt fotosyntes, indtil det pludseligt bliver varmt (termometer på AV) og algen forlader korallen hvorefter den bliver grå enten ved noget smart med lys eller kostumer. Korallen fortryder nu dens beslutning om at opgive sine motor-neuroner for selvom det er hårdt at tænke på verdens problemer er det alligevel godt at kunne flytte sig når havet brænder på, så at sige. Den kan også slutte med at korallen ædes af en tornekronesøstjerne
\url{https://en.wikipedia.org/wiki/Crown-of-thorns_starfish}
Konceptet kører lidt over Pinocchio, hvor Planulaen er Pinocchio himself, mens Motorneuronet er Jesper Fårekylling og Algen er ham der æseldrengen på den der forlystelse. Tornekrone søstjernen er så Monstrum.


\begin{sketch}

\scene{Beskrivelse}
\scene{Sort lys ned}

\scene spot på p og m støder sammen på scenen

\says{p} Jeg lever! Hov, hve\ldots hvem er du?

\says{m} Jeg er din samvittighed Jesper Motorneuron!

\says{p} nå, spændende, hvad kan sådan en motorneurons samvittighed bruges til?

\says{m} Jeg kan hjælpe dig med at komme omkring, og det uden at gøre noget forkert! 

\says{p} aah, det er jeg vel nok glad for at du vil hjælpe mig med

\says{p} skal vi så ikke se at komme afsted? Lad os svømme den vej!(peger imod publikum)

\says{m} argh, den vej skal vi passe på dem, den er fyldt med farlige mennesker- de har krig, forurening og druk som vi slet ikke vil være en del af

\says{p} nå, ja øh, hvad så med den vej?

\says{m} nah, det går ikke, det kina, der blir vi bare spist eller blir filmstjerner, de er syge i hovedet

\says{p} ej, nå, hva så med den vej?

\says{m} ej, det Amager, de humanister, det de værste af ALLE

\says{p} ej, der kan da ikke være noget galt i alle retninger, hva så med den der? 

\says{m} Der er Thailand, der kan vi vær\ldots Nah, og dog, den regering har vidst ret meget korruption og ladyboys

\says{p} ej forhelvet! Så gider jeg da slet ikke bevæge mig overhovedet!

\says{A} var der nogen der sagde stationær!

\says{p} neeej\ldots Det var der ikke, måske lidt, men ikke lige det ord

\says{A} visse vasse, jeg er Coralline, og jeg er du ved ret sej og sådan- jeg kan nemlig fotosyntisere 

\says{p} fotysontisere? det faktisk pænt sejt ja

\says{A} jaa, så jeg kan faktisk gøre den der motorneuron total overflødig du, du skal bare lade mig bo i dig lidt ik

\says{p} nå, spændende 

\says{A} ja ja ik, så laver jeg fotosynthese af al den der dejlige lys og så kan du bare chille her og ikke tænke på noget ik, så klarer jeg bare det hele

\says{p} orv det lyder faktisk ret nice

\says{m} ej hov, hør nu her, hvad med global opvarmning, du blir nødt til at\ldots

\says{A} ha ha, ja det jo intet problem, jeg laver jo fotosynthese så jeg eeeelsker mere lys, \scene{intenst lys på scene} stol du bare på mig, så tager jeg mig af det hele- du skal bare lige\ldots Du ved\ldots Æde ham/hende der        

\says{p} men det jo min\ldots

\says{A} ja ja, jeg ved det, men gør det nu bare lige ik, så tager jeg mig af aaalt det andet 

\says{p} oh\ldots Okay

\says{M} du vil fortryde det, jeg er din samvittighed, lyt nu!

\act{p æder m}

\says{A} feedest, så flytter jeg bare ind og går igang

\says{p} hehe, velkommen til roomie

\says{A} her fandme fedt ik, så sætter vi os bare her på bunden og så laver vi en stor flot koral du, blir awesome!

\says{p} yeah!

\act{p og a bygger en koral mandehybel, props tages ind} 

\says{AV} \act 30 sekunders musik

\says{p} Jubi jeg vandt igen! Vi har det sgu ret da ret fedt, her er sgu da fandeme fedt. Skal vi ikke taget et ekstra spil bro.

\says{a} Det ved jeg ikke lige\ldots men her er da sådan lidt varmt \scene{varmt lys}

\says{p} men, sagde du ikke det var godt med en masse sol?

\says{a} tjo, men det her er sådan lidt i overkanten 

\says{p} men men du sagde

\says{a} jeg ved hvad jeg sagde, men det her sgu ikke lige det jeg havde tænkt, fuck det her spil mand. Ses på revet.

\act{a pifter og to ninjaer kommer ind og fjerner alle props}

\says{p} nej vent nej, jeg kan\ldots Jeg\ldots

\act{a svømmer væk} 

\says{p} nå, hvad gør jeg nu, nu har jeg hverken alge eller motorneuron, jeg kan ikke bevæge mig og jeg har ingen til at give mig energi

\says{p} ååååh neeej, neeej, gid jeg aldrig havde spist min motorneuron, forbandet tag dig din dumme alge!!!

\act{m kommer ind som spøgelse}

\says{kor} uhhhh

\says{m} Nå så sidder du der, ja med mig havde vi jo kunne svømme til køligere steder, men nej, du ville hellere hænge ud her med ham algen\ldots

\says{p} Undskyld, jeg har lært min lektie. Kan du ikke komme tilbage? Jeg vil bare gerne være en lille planula igen. Er det et\ldots et stjerneskud? 

\scene{Spot kører rundt og ender på cot}

\says{m} Nej, det er en søstjerne. Mere specifikt er dets havets plage og den globaleopvarmnings prins Tornekronen!

\scene{spot følger cot}

\says{cot} Var der nogen der sagde stationær nomnomnomnom

\says{p} aaaaargh

\scene cot spiser p, meget langsomt

\sings{kor} Når du ser en stjerne stor, er der ingen kære mor, nu vil alle stjerners herre æde dig\ldots

\scene{Sort lys op.}

\end{sketch}
\end{document}
