\documentclass[a4paper,11pt]{article}

\usepackage{revy}
\usepackage[utf8]{inputenc}
\usepackage[T1]{fontenc}
\usepackage[danish]{babel}

\revyname{Biorevy}
\revyyear{2017}
% HUSK AT OPDATERE VERSIONSNUMMER
\version{0.1}
\eta{2.23 min}
\status{Færdig}
\responsible{tex@biorevy.dk}

\title{Labchikane}
\author{Isabella}

\begin{document}
\maketitle

\begin{roles}
    \role{Stud}[Isabella] Ung studine
    \role{Prof}[Aske] Klam underviser
    \role{Info}[Caroline] Informant
    \role{Ninja}[Jeppe]
\end{roles}

\begin{props}
    \prop{Stol}[Person, der skaffer] Stol
    \prop{Bord}[Person, der skaffer] Det er der ingen i Biorevy, der ved hvad er
    \prop{Stereolup}[Person, der skaffer] Stereolup
\end{props}


\begin{sketch}

\scene{Beskrivelse}
\scene{Sort lys ned}

\scene{Studinen sidder ved stereoluppet og kigger}
\scene{Underviseren kommer ind på scenen og kigger a over skulderen}

\says{Underv} Det er da godt nok 2 store flotte alger du har der!
\says{Stud} Tusind tak! Men jeg tror vidst nok det er en Volvox, og som du nævnte i sidste uge består den af mange små individer.

\says{Underv} Lad mig se.
\scene{Underviseren kigget i luppet}
\says{Underv} Ja, det er sørme korrekt.
\act{Underviseren rejser sig}
\says{Underv} Ja Ja Ja, to STORE flotte RUNDE bløde spændstige velformede, håndfaste alger Kan du fortælle mig lidt om dem? f.eks. hvordan de har det med lys? 

\scene{Studinen svarer meget stolt og med brystet fremme}
\says{Stud} De vil meget gerne ud i lyset!

\says{Underv} Skal vi så ikke åbne lidt op så de får lidt mere lys og vi bedre kan se dem?
\act{Underviseren indikerer mens dette siges til at pigens trøje skal åbnes}

\says{Stud} Sikken en god ide!
\scene{a skruer op for lampen på luppet}

\says{Underv} Hvordan er det så de bevæger sig mod lyset?

\says{Stud} Der er de her punkter på den ene side der er meget sensitive -- ikke?
\scene{Studinen for peget ud fra sine bryster mens dette siges}

\says{Underv} Åh ja

\says{Stud} Og de sender vidst nok signaler til resten af organismen 
\says{Underv} ja om at komme. Du er godt nok klog
\scene{Studinen smiler op mod Underviseren der har fokus på helt andre ting}

\says{Underv} Hvad kan du så sige om deres kønnede formering?
\act{Underviseren tager fast i hendes skuldere mens hænder glider ned mod brystet}

\scene{Studinen og Underviseren fryser, lyset dæmpes}
\scene{Infomationsgiveren kommer ind på scenen i spot}

\says{Info} Labchikane finder sted overtalt, selv hvor du mindst venter det og har ingen plads i forskermiljøet. Hvis at du oplever labchikane så rapporter det til din lokale studieadministration/greenlighthouse/help desk, så KU kan opnå et sundere studie og forsker miljø.  \act{kunstpause} men husk du vil jo gerne have den PhD


\scene{lys ned}


\textbf{Backup afslutning}
\says{Stud} Og så får jeg tolv ik?
\says{Underv} JA JA!
\scene{Stud falder på knæ}
\says{Underv} NEEEEEEEEEEEEEEEJ



\scene{Sort lys op.}

\end{sketch}
\end{document}
