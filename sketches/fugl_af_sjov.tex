\documentclass[a4paper,11pt]{article}

\usepackage{revy}
\usepackage[utf8]{inputenc}
\usepackage[T1]{fontenc}
\usepackage[danish]{babel}

\revyname{Biorevy}
\revyyear{2017}
% HUSK AT OPDATERE VERSIONSNUMMER
\version{0.1}
\eta{3.17 min}
\status{Færdig --\emph{fugl}endt}
\responsible{ordspild@biorevy.dk}

\title{Fugl af sjov}
\author{Tristan, Nina -- tidl. Valdbjørn \& Karla}

\begin{document}
\maketitle

\begin{roles}
    \role{Falk}[Aske] Falk. Midaldrende gangster-agtig/ung-med-de-unge type.
    \role{Peli}[Nicolai] Peli-can't. Midaldrende gangster-agtig/ung-med-de-unge-type.
    \role{AV}[AV] AV med billeder af omtalte fugle.
\end{roles}

\begin{props}
    \prop{Fuglebilleder}[Person, der skaffer] AV med billeder af omtalte fugle.
\end{props}

\textbf{Kommentarer/kontekst:}
Ordspild. De kan laves som to ornitologer, men vi tænker dem mere som to middelaldrende mænd der forsøger at være unge og derfor har købt noget gangster tøj og prøver at genskabe deres fordums coolness. Gerne med AV med billede af fuglene der bliver nævnt, når de bliver nævnt. Peli er sur, mens Falk er jovial.
Ubrugte ordspil
Duenf
Næb
Skarv
Aves


\begin{sketch}

\scene{Beskrivelse}
\scene{Sort lys ned}

\says{P} god aften \textit{Falk}
\says{F} Hey \textit{Peli, kan} du ikke sige ordentligt hej, jeg er kommet fra mågeløv?
\act{P krammer F meget voldsomt og forsøger at lave et bro-shake af en art, som F ikke helt fanger}
\says{P} Kan du ikke lige \textit{fjerne} dig lidt? 
\says{F} Puha \textit{Tjur} hej hvor det går\ldots er du ude af \textit{traning} eller hvad?
\says{P} Njaeee jeg er bare bange for du \textit{ugler} min \textit{måg-hawk}
\act{P tager sig beskyttende til håret}
\says{F} Ja undskyld, jeg gav den vist lidt meget \textit{gås} i det kram
\says{P} ja, det gjorde du, men \textit{svalene} var det nu alligevel
\says{F} Men helt ærligt, jeg kunne jo mærke at du har \textit{glente} vores bro-shake..Skal jeg \textit{ægge} lige lære dig det igen
\says{P} Jo altså \textit{hjejle} ver godt nok efter mottoet \textit{grib} chancen og få det \textit{vinget} af’, men faktisk kan jeg ikke, da jeg har en \textit{Skade}, som jeg fik for nylig.
\says{F} Var det din \textit{stårker} der gav dig den eller er der en \textit{anden} grund?
\says{P} Nej faktisk er det fordi jeg drak mig \textit{edderfugl} i sidste weekend
\says{F} Ja, jeg hører godt nok at du var ret \textit{Flyvende}. Var du på \textit{Kro?}
\says{P} Hov hov, hvem er det der har \textit{snadret?!} Det \textit{ravner} ikke folk hvor jeg drikker mig \textit{fugl}
\says{F} Jeg vil da næsten mene, at jeg som din \textit{crow}, har \textit{krage} på at høre det. Men okay, vi behøver heller ikke \textit{træne} det bro-shake, hvis du ikke vil..
\act{P holder hånden op til øret og tysser med den anden hånd}
\says{P} shhhh!
\says{F} huuuh?
\says{P} Kan \textit{due} også \textit{høger} nogen der \textit{Sjagger}
\says{F} Neeej, jeg tror bare, det er \textit{Lærke} der Råger. 
\says{P} Er det ikke hende med \textit{peacocken?}
\says{F} Ja, lad os håbe hun ikke kommer herhen den \textit{finke}
\act{F spejder ud over publikum for “at holde øje med Lærke”} 
\says{P} Ja okay, det er måske bare mig der ikke kan høre et \textit{PIP} 
\says{F} Haha nej, og du kan sgu heller ikke se ordentligt efter du fik konstateret \textit{grå stær} 
\says{P} Så sandt så sandt, sikke en \textit{redelighed}
\says{F} Måske havde du for meget træsprit i lomme-\textit{lærken}
\says{P} Du er godt nok \textit{rap}-kæftet i dag, hva’?!
\says{F} Ja, jeg tager da min T\textit{ørn}. Men du \textit{Piber} da vist også lidt meget.
\says{P} Orh hold da kæft man, kan du ikke bare \textit{Skråpe} af!
\says{F} Jo da, men jeg tager dig med bro. Jeg skal mødes med min ven, \textit{Munk}. Kom lad os \textit{lunde} på \textit{den gyldne måge}

\act{P pejer op på lyset som slukker/bliver sort}



\scene{Sort lys op.}

\end{sketch}
\end{document}
