\documentclass[a4paper,11pt]{article}

\usepackage{revy}
\usepackage[utf8]{inputenc}
\usepackage[T1]{fontenc}
\usepackage[danish]{babel}

\revyname{Biorevy}
\revyyear{2017}
% HUSK AT OPDATERE VERSIONSNUMMER
\version{0.2}
\eta{0.82 sek}
\status{Færdig}
\responsible{En revyst}

\title{Når mor ringer (3:3)}
\author{Laurids}

\begin{document}
\maketitle

\begin{roles}
    \role{Rus}[Aske] Biorus
    \role{Mor}[Miriam] Biorussens mor
\end{roles}

\begin{props}
    \prop{Rushat}[Person, der skaffer] Rushat m. propel
    \prop{Uldtrøje}[Person, der skaffer] Uldtrøje til russen.
    \prop{Gummistøvler}[Person, der skaffer] Gummistøvler til russen.
    \prop{Telefoner}[Person, der skaffer] 2 telefoner
    \prop{Mor}[Person, der skaffer] Mor-kostume. Forklæde m.v.
\end{props}

\textbf{Kommentarer/kontekst:}
En lille sketch om hvordan det er når mor ringer til den unge BioRUS, der bare elsker sit fag, og hvad mor egentlig er interesseret i.




\begin{sketch}

\begin{center}
\textbf{Del 3: Planter}
\end{center}
\scene Lys op.
\scene Mor og RUS står i hver sin side af scenen med spot på sig. RUS er iført strikket uldtrøje og gummistøvler \emph{uden RUShat.}
\scene Telefonen ringer, Rus'en tager telefonen
\says{Rus} Hej mor.
\says{Mor} Hej skat! Nu er det jo så lang tid siden vi sidst har snakket sammen.
\says{Rus} Du ringede sidste uge. \emph{Igen!}
\says{Mor} Hvordan går det så på dit studie?
\says{Rus} Ja, jeg har lært noget om nogen planter. Det er vildt spændende! Plan-\act{Mor afbryder -- hun har fået nok af alle de biolog-ting!}
\says{Mor} DÈT lyder rigtig nok spændende!
\says{Rus} Det er det virkelig også.
\scene Tavshed
\says{Mor} Er du så homoseksuel måske?


\scene{Sort lys op.}

\end{sketch}
\end{document}
