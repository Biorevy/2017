\documentclass[a4paper,11pt]{article}

\usepackage{revy}
\usepackage[utf8]{inputenc}
\usepackage[T1]{fontenc}
\usepackage[danish]{babel}

\revyname{Biorevy}
\revyyear{2017}
% HUSK AT OPDATERE VERSIONSNUMMER
\version{0.1}
\eta{5.85 min}
\status{Idéfase?}
\responsible{tex@biorevy.dk}

\title{Administrationen der gik galt}
\author{Tristan, Laurids, Jonathan, Nathalia og Nina (med ide fra Søren)}

\begin{document}
\maketitle

\begin{roles}
    \role{Kont1}[Veronica] Kontorperson
    \role{Kont2}[Emil] Kontorperson
    \role{Stud}[Nicolai] Studerende
    \role{Helge}[Tristan] Helge Roooo
    \role{Ninja}[Camilla]
    \role{Ninja}[Laurids] 
\end{roles}

\begin{props}
    \prop{Computer}[Person, der skaffer] Stor, gammeldags computer
    \prop{Telefon}[Person, der skaffer] Telefon, gerne gammeldaws
    \prop{Bord}[Person, der skaffer] Et Bord
\end{props}


\begin{sketch}

\scene \textbf{Kommentarer/kontekst fra forfatterne}
Den daglige ledelse af administrationen og hvordan det hele bare går galt.\\
Inspireret af https://www.youtube.com/watch?v=DOWO4gq-whg
Det der sker i sketchen er så en studerende og en administrations type, de står og taler om et problem den studerende har. De forsøger at løse problemet og mens de gør det så går der en masse sceneteknisk galt og de forsøger at løse problemet med dumme løsninger der gør grin med administrationen. Generelt skal vi have lavet et generelt plot, der går galt. 
\emph{Generelt plot der ikke går galt:} En person kommer ind og er glad, for han skal have hjælp af den hjælpsomme studieadministration. Han forklarer han skal have merit overført. De kigger på hans fag og diskuterer hvilke der kan blive merit overført og hvilke der ikke kan. Der er et problem med et af fagene, derfor sender de bud efter studielederen der ankommer med det samme. Han forklarer at det er et  projekt udenfor kursusregi, der godt kan blive merit overført til det nye studie og dermed konkluderes en succesfuld dag i administrationen. Ellers gør vi det med mysterie koncept: Mysteriet om Den Forsvundne Merit



\scene På scenen står et bord med computer og papirer. 2 kontorpersoner står/sidder og skriver ``vigtige'' ting på computeren. Det hele er ordentligt og pænt og super strømlinet.
\scene{Sort lys ned}
\says{K1} Velkommen til KU’s nye administration. Denne velsmurte effektive maskine af lidelse - øhh nej ledelse, hvor der aldrig går noget galt. Den diktatoris- øhm - bureaukratiske proces er aldrig en hindring for studerende når de skal have hjælp til det administrative i deres studie. Velkommen til Mysteriet om den forsvundne merit. 
\act Ved ordet ``bureaukratiske'' kigger Kont1 i håndfladen for ``at læse sin replik''. Sætter sig så på stolen ved bordet. Den anden stol er tom
\says{Stud} Hej, jeg skal have overført noget merit, men det som om hjemmesiden ikke gider gøre som jeg siger den skal
\says{Kont1} Ehj\ldots Hej studerende, det vil vi meget gerne hjælpe dig med din tvangsmeri- ehm, merit; det er jo det vi er her for!
\says{Stud} Jo, nu skal du se, du kan se her på min\ldots Øh\ldots Computer\ldots Hvor var det nu?
\act{ninja løber ind på scenen, vælter, kejtet rejser sig og giver stor gammeldags computer, lister ud igen efter(rigtig kikset)}
\says{Stud} øhm ja\ldots Som du kan se her, så har jeg læst biokemi før og vil i den forbindelse ikke rigtig have videnskabsteori en gang til\ldots Men det er som om at mine point ikke er blevet registreret, så jeg har en karakter, men der er ingen ECTS point.
\says{Kont1} Det var da underligt, det er den afdeling min kære kollega har styr på…...
\act Stud og Kont1 kigger på den tomme stol
\says{Kont1} DET ER DEN AFDELING MIN KÆRE KOLLEGA HAR STYR PÅ!
\act{Kont1 kigger utålmodigt hen mod bagtæppet} 
\says{Stud} Det var da godt øhhh at han har det.
\says{Kont1} Ja men det har han da… jo... Meget godt styr på alt.. Der er ikke det han ikke har STYR PÅ!
\act{Kont1} lister sig under sin replik sidelæns og akavet hen mod bagtæppet og hiver det til side for at se hvor Kont2 er. Kont2 står og snaver/flirter (hvad end vi kan få dem til) med en fra en tidligere sketch/dans (helst en med et fjollet/grimt kostume. Efter den første forskrækkelse hiver Kont1 hurtigt tæppet for igen og lister sig tilbage til stolen. Kont2 kommer efter nogle sekunder ind på scenen (gerne med et læbestiftkys på kinden og smider evt en BH ud til publikum) lidt forvirret meget glad for at være med
\says{Kont2} HEJ MOR! .
\act{Kont2} vinker til publikum og Kont1 og Stud kigger irriteret på ham)
\says{Kont2} øhm, ja øh, jo nu skal jeg lige kigge, ja det da noget fjollet noget, jeg kan jo se at du har en karakter, men fordi alt nu skal være så fjollet, eller nej, sikkert, så må vi lige spørge din underviser, om det nu er gået som du siger.
\says{Helge} Hej, jeg er Helge! Nå, hvad kan jeg…..(bliver afbrudt)
\act{Helge bliver gennet ud af den studerende, fordi han er kommet for tidligt ind på scenen}
\says{Kont2} Så må vi lige spørge din underviser, om det nu er gået som du siger.
\says{Kont1} Godt, jeg ringer lige og hører om han kan komme
\act{Kont1 tager telefonen og ringer}
\says{Kont1} Hej Helge. Har du mulighed for at komme over til..stu-di-ad-min-stra-tionen med det samme? Det var godt.
\act{Kont1 kigger ved ordet ‘studieadministrationen i håndfladen igen og udtaler det forkert}
\says{Helge} Hej, jeg er Helge! Nå, hvad kan jeg gøre for jer (med dybere stemme)
\says{Kont1} Ja du skal validere om denne studerende har taget det her kursus, det burde din hemmelige adgangskode kunne finde frem til
\says{Helge} Jo.. øh, ja den har jeg øh..
\act{skuler ud mod sidetæppet og sætter i løb hen for at hente en stor kuffert og kår helt cool tilbage til bordet med kufferten som om intet er hændt. I dovenskab smækker han kufferten op på bordet, men misser eller noget og slår Kont1 ud. Hvis vi kan få kufferten til at ryge midt over ville det være nice}
\says{Kont2} Øhm, har du fundet resultaterne? (henvendt til bevidstløse Kont1)
\says{Stud} Kan du se hvad problemet er? (også henvendt til bevidstløse Kont1)
\act{Kont2 og Stud kigger på Kont1}
\says{Stud} Så ja det ser okay ud..eller..? Hold da op hvor ser du øhhh koncentreret ud!
\says{Kont2}Ja man fristes næsten til at spørge OM DU ER OKAY.
\act{Band} Begynder at spille en sang og Stud og Kont2 prøver at tale hen over musikken
\says{Stud} kan du så se at jeg har klaret kurset! Hov, hvorfor går du?
\says{Kont2} Nu undersøger hun det i et andet… øh kontor
\act{Kont1 ligger stadig på gulvet og der kommer nogle ninjaer ind og forsøger at trække Kont1 ud, Helge hjælper eventuelt}
\says{Stud} Kan du se om jeg har klaret kurset?
\says{Kont2} Ja det må jeg sige! Flot klaret! Vi overfører pointene med det samme!
\says{Helge} Godt vi fik\ldots Klaret den uredelighed 
\act{Band stopper med at spille fordi Helge under sin replik trækker et stik ud eller tager deres instrumenter eller eller eller}
\says{Stud} så kan jeg bare begynde min nye bachelor her ik’? 
\says{Kont2} nåh, nej det kan du ikke, nu kan vi jo se at du allerede har en bachelor i biokemi, så kan du jo ikke få en til -- Farvel!
\act{Stud} Ser opgivende mod publikum, mens Kont2 vinker farvel med sin ene hånd og løfter Kont1’s hånd og vinker farvel
\scene{Sort lys op.}

\end{sketch}
\end{document}
