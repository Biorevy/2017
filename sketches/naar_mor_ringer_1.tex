\documentclass[a4paper,11pt]{article}

\usepackage{revy}
\usepackage[utf8]{inputenc}
\usepackage[T1]{fontenc}
\usepackage[danish]{babel}

\revyname{Biorevy}
\revyyear{2017}
% HUSK AT OPDATERE VERSIONSNUMMER
\version{0.2}
\eta{0.83 min}
\status{Færdig}
\responsible{En revyst}

\title{Når mor ringer (1:3)}
\author{Laurids}

\begin{document}
\maketitle

\begin{roles}
    \role{Rus}[Aske] Biorus
    \role{Mor}[Miriam] Biorussens mor
\end{roles}

\begin{props}
    \prop{Rushat}[Person, der skaffer] Rushat m. propel
    \prop{Uldtrøje}[Person, der skaffer] Uldtrøje til russen.
    \prop{Gummistøvler}[Person, der skaffer] Gummistøvler til russen.
    \prop{Telefoner}[Person, der skaffer] 2 telefoner
    \prop{Mor}[Person, der skaffer] Mor-kostume. Forklæde m.v.
\end{props}

\textbf{Kommentarer/kontekst:}
En lille sketch om hvordan det er når mor ringer til den unge BioRUS, der bare elsker sit fag, og hvad mor egentlig er interesseret i.




\begin{sketch}

\scene{Beskrivelse}
\scene{Sort lys ned}
\begin{center}

\textbf{Del 1: Vandmanden}
\end{center}

\scene Lys op. RUS er iført sin RUShat.
\scene Mor og RUS står i hver sin side af scenen med spot på sig. 
\scene Telefonen ringer, Rus'en tager telefonen
\says{Rus} Hej mor.
\says{Mor} Hej skat! Nu er det jo så lang tid siden vi sidst har snakket sammen.
\says{Rus} Du ringede sidste uge.
\says{Mor} Hvordan går det så på dit studie? Er det godt?
\says{Rus} Ja, nu skal du høre! Vi har lige lært om ª\textit{Cnidaria} og deres livscyklusser! Først, så starter de som planula larve, hvor at de så derefter bliver til en fastsiddende polyp der formerer sig aseksuelt via knopskydning. Og så har de jo deres kønnede medusa stadie hv\ldots
\says{Mor}Medusa? Er der ikke hende med slangerne på hovedet?!
\says{Rus}[suk!]Nej, mooor, det er noget helt andet! 
\says{Mor} Det lyder rigtig nok spændende.
\says{Rus} Det er det virkelig også.
\says{Mor} Har du så fundet dig en sød pige?
\says{Rus} Ahrj, mor \ldots Parringssæsonen er jo kun lige begyndt!
Sort lys. 
Slut på Del 1.


\scene{Sort lys op.}

\end{sketch}
\end{document}
