\documentclass[a4paper,11pt]{article}

\usepackage{revy}
\usepackage[utf8]{inputenc}
\usepackage[T1]{fontenc}
\usepackage[danish]{babel}

\revyname{Biorevy}
\revyyear{2017}
% HUSK AT OPDATERE VERSIONSNUMMER
\version{0.1}
\eta{5.70 min}
\status{Ikke færdig}
\responsible{tex@biorevy.dk}

\title{Den akademiske borgerkrig}
\author{Tristan, Jonathan}

\begin{document}
\maketitle

\begin{roles}
    \role{Fys}[Emil] Fysiker
    \role{Mat}[Jakob] Matematiker
    \role{Dat}[Nathalia] Datalog
    \role{Bio}[Tristan] Biolog
\end{roles}

\begin{props}
    \prop{Rekvisit}[Person, der skaffer] Rekvisitforklaring
\end{props}

\textbf{Kommentarer/kontekst fra forfatterne}
Intro alla Kaspar Colling Nielsens den danske borgerkrig. Den sidste statskundskaber er blevet henrettet og en fysiker, en matematiker og en datalog diskuterer hvem som skal overtage magten. De stoler ikke på hinanden, men heldigvis træder Biologen ind på scenen. Han bliver valgt som et kompromis for hvad skade kan han gøre. Han tager beslutninger, i starten er de ufarlige og egentlig ret nice (eksempelvis ordner miljøet, stopper sult etc.) senere når der kommer problemer bliver de mere og mere nazi. Ting i parantes efter binær for datalogen kan nok smides op på AV som oversættelser. 


\begin{sketch}

\scene{Beskrivelse}
\scene{Sort lys ned}
\scene Hvid dug med ninja og lys bagved, det ligner en person får hugget hovedet af i en guillotine
\scene{lys op}
\scene Fys, Mat og Dat træder ind på scenen foran lagnet
\says{Fys} Sejren er hvor! Den sidste statskundskaber er fældet!
\says{Dat} 01001010011101010110100001110101 (juhu)
\says{Fys} Som den nye præsident\ldots
\says{Dat} 01001000011101100110 (Hvad?)
\says{Mat} Enig, jeg er helt klart bedst kvalificeret!
\says{Dat} 011000010111001001101000 (Arh)
\says{Mat} Hvem foreslår du da? Fysikeren, han vil jo bare have en computer der kører på atomkraft
\says{Dat} 010001000110000101 (det hedder altså en datamat)
\says{Fys} Ikke bare computere, alt ja selv din tamagoji...  
\scene Mat, Dat og Fys skændes -- imens kommer Biologen ind på scenen. Bio serverer mad for Dat, Fys og Mat
\says{Bio} Her mine naturvidenskabelige venner, jeg har lavet noget I kan styrke jer på efter den lange, blodige og trættende krig.
\says{Mat} Aha! Biologen! Han kan være den nye præsident. Han kan vel alligevel ikke gøre nogen skade. Han ved jo alligevel ikke noget om matematik
\says{Bio} Betyder det jeg ikke skal over på Jobcentret i dag?
\says{Fys} Ja  han går vel an. Nu kan du gå ud og bestemme for os, vi tror du kan finde den bedste løsning.
\says{Mat} Du kan starte med at rydde op efter den her krig\ldots
\scene{Biologen går ud af scenen, meget glad. Fys og Mat sætter sig. Dat trækker computer frem. AV}
\says{AV}\act{10 dage senere *på datasprog log(100)}
\says{Fys} Nå lad os se hvad han har udrettet
\says{Mat} Han er jo lige startet han kan vel ikke have nået så meget\ldots
\says{Dat} (tager sin datamat frem og kigger) 01001001010101010010 (han har reddet Amager Fælled)
\says{Mat} Det kan man jo også
\says{Dat} 1010100001111100001 (nu har han kureret cancer)
\says{Fys} Hvor vildt, hvordan?
\says{Dat} 1000001000100 (han ændret lidt på funding systemet)
\says{Mat} Smart, hvorfor har ingen af os tænkt på det? Hvis der er penge til lortet så dur, det åbenbart. Det er jo omvendt af hvordan statskundskaberne plejede at styre landet. Jeg kan lide det!
\says{Dat} 100001001100011111 (Nu har han gjort en ende på energi-krisen og hungersnød)
\says{Fys} Sejt
\says{Dat} 1001010101011111 (han har fjernet det danske kongehus)
\says{Mat} Nå, det var måske på tide, men hvorfor er det en prioritet?
\says{Dat} 111001110001100101 (nu har han indsat sig selv som alpha individ)
\says{Fys} Ligesom partiklen? Hvor mærkeligt
\says{Dat} 110010101011010101 (nu har han løst finanskrisen)
\says{Mat} Hvad? Jeg troede ikke han kunne regne?
\says{Dat} 1000101010101010000 (han har gjort det ved at indføre naturlig-selektion på ældreplejen) 
\says{Mat} Guys\ldots
\says{Dat} 1001010101010001 (nu har han nedlagt hele medicinal industrien)
\says{Fys} Det her kan ikke fortsætte
\says{Mat} Vi finder guillotinen frem igen…
\scene{B træder ind på scenen}
\says{Bio} Hej venner det var godt nok lærerigt, fedt med noget erhverserfaring. Nu har jeg noget at skrive på mit CV. En gang skal jo være den første, som man siger høhø.
\says{Fys} Ja vi har jo fulgt lidt med, det var nogle interessante valg du lavede her til sidst...  
\says{Mat} Så hvis du bare lige. (peger hen mod den modsatte side af scenen, B går derhen mens D, F og M gør sig klar til angreb). 
\says{Bio} Ja jeg havde jo fulgt lidt med i jeres "alliance" mod statskundskaberne, det var meget modigt ja endda ædelt af jer at I bortselekterede de udskud. Jeg observerede, noterede. De vigtige ting om jeres fremskridt og jeres masse. Jeg gik ud i mosen igem hvor jeg havde gemt min seneste opdagelse, en ny og forbedret S. Pyogenes. Jeg håber maden smagte godt, den burde have brudt gennem væggen til jeres hjertekammer omkring nu, I kunne alligevel aldrig bruges til så meget andet end næringsomsætning. 
\says{Fys} Det kan godt være du fik ram på os, men vi tager dig med os På ham!
\says{Fys, Dat og Mat} arhhh
\act{Fys, Dat og Mat falder døde om i kramper. Bio drikker et whisky-glas med snaps}
\scene {“evil morty sang” i baggrunden https://www.youtube.com/watch?v=Me6\_L5vMWPE}
\scene{Sort lys op.}
\end{sketch}
\end{document}
