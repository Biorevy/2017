\documentclass[a4paper,11pt]{article}

\usepackage{revy}
\usepackage[utf8]{inputenc}
\usepackage[T1]{fontenc}
\usepackage[danish]{babel}

\revyname{Biorevy}
\revyyear{2017}
% HUSK AT OPDATERE VERSIONSNUMMER
\version{0.1}
\eta{4.68 min}
\status{Færdig}
\responsible{tex@biroevy.dk}

\title{Akademikeren med handskerne}
\author{Valdbjørn}

\begin{document}
\maketitle

\begin{roles}
    \role{Aka}[Veronica] Akademikeren
    \role{Fort}[Caroline] Fortæller-H.C. Andersen-type
    \role{Prof}[Mads] Professor
    \role{Ninja}[Jonathan]
\end{roles}

\begin{props}
    \prop{Handsker}[Person, der skaffer] Store handsker
    \prop{Luffer}[Person, der skaffer] \ldots endnu større luffer
    \prop{Kup}[Person, der skaffer] Kaffekup
\end{props}


\begin{sketch}

\scene{Beskrivelse}
\scene{Sort lys ned}

\says{Fort} Der var engang i fordums tid en kløgtig professor og han ledte efter en akademiker assistent så ægte og vis. Men dog voksede disse ikke på træerne så grønne.

\says{Prof} Åh ak åhh ve hvor skal jeg dog lede?!

\says{Fort} Den gamle professor rejste hele det store biocenter rundt fra kælder til kvist fra bygning, 4 til 2. studerene var der i store mængder men ganske grælt stod det til med akademikerne. Altid var der skavanker med de studerende og ingen ide havde Professoren om hvad der var galt.

\says{Prof} Åh nej jeg arme stakkel ikke en eneste fin og ægte forsker er der tilstede. Min gamle krop er så ganske tømt for kræfter.

\says{Fort} Den gamle professor drog således til kantinen, for at samle mod og energi til endnu en rejse

\says{Prof} Se nu er jeg i kantinen og det er ganske vidst. Må jeg bede om en kop forfriskende bønne bryg til min åh så afkræftede krop.

\says{Aka} Mener du kaffe?? det er ovre i automaten

\says{Fort} Men den gamle professor forstod hverken for eller bag af hvad ungmøen sagde.

\says{Prof} jeg forsår jo hverken for eller bag af hvad du siger. Lidt bønne bryg og helst med hast.

\says{Aka} (ruller med øjnene) okay da, (går over og hælder en kop kaffe op) skide forskere (sagt til sig selv)

\act{A hælder kaffe op og får en brændvabel af det}

\says{Aka} Av for helvede! Lortevarm kaffe. Nu får jeg en vabel!

\says{Fort} Da indså den gamle professor at han havde fundet en rigtig akademiker

\says{Prof} Sikke dog nogle fine og sarte hænder, kunne dette kan være en ægte akademiker, som jeg? (sagt til sig selv).

\says{Fort} Nu var gode råde dyre, hvordan kunne han dog narre den lovende akademiker ud på den afgørende test?

\says{Prof} hvordan får jeg hende, dog narret hen på mit forsker kammer?

\says{Prof}[Sagt til sig selv] Hvilken ide!

\says{Prof} Hvad vil du sige til at tjene en skilling eller to ved at være en aldrende herre til hjælp?

\says{Aka} tjoo hvad skal du have hjælp med? der er alligevel ikke andre i kantinen.

\says{Prof} thi det er blot et lille søm der skal ordnes og til det har jeg hverken styrke eller kendskab til at klare selv.

\says{Aka} en lille skilling siger du? Penge mangler jeg altid. Vis mig sømmet! Forresten... hvad er et søm?

\says{Prof} følg du blot med til mit forskerkammer.

\says{Fort} Og således drog professoren og han aspirende akademiker assisten mod forskerkammeret. Der fik de straks øje på søm og hammer.

\scene A \& P følges ad til forskerkammeret(kontoret)

\says{Prof} Her er både søm og hammer.

\says{Aka} Skal jeg slå det der i? Med den der? Det kan jeg da ikke uden beskyttelse.

\says{Prof} Perfekt! Her er handsker, tag du og ifør dig disse  fyrreogtyve plastik-handsker og derefter fyrreogtyve af dem til varm!

\says{Aka}[skeptisk] tjaaah, <fragment, consider revising>

\act{A slår på sømmet.}

\says{Aka} Av for søren, jeg tror minsandten at jeg har slidt al huden af mine håndflader.

\says{Fort} Da vidste den gamle professor at han i sandhed havde fundet den rette assisstent/studenrende/akademiker.

\says{Prof} Du er i sandhed en ægte akademiker, Kun en akademiker har så sarte hænder at hun kan slides op selvom hun har fyrreogtyve plastikhandsker og fyrretyve af dem til varme på!.

\says{Fort} Og de levede lykkeligt til tundingens ende.


\scene{Sort lys op.}

\end{sketch}
\end{document}
