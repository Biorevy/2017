\documentclass[a4paper,11pt]{article}

\usepackage{revy}
\usepackage[utf8]{inputenc}
\usepackage[T1]{fontenc}
\usepackage[danish]{babel}

\revyname{Biorevy}
\revyyear{2017}
% HUSK AT OPDATERE VERSIONSNUMMER
\version{0.1}
\eta{5.25 min}
\status{Færdish}
\responsible{tex@biorevy.dk}

\title{En truet art}
\author{Laurids}

\begin{document}
\maketitle

\begin{roles}
    \role{DaAtt}[Jeppe] ``David Attenberg''/Vært
    \role{Hum1}[Jakob] Humanist 1/Klassisk arkæolog
    \role{Hum2}[Nicolai] Humanist 2/Litteraturhistoriker
    \role{Hum3}[Emil] Humanist 3/Teolog
    \role{Ninja}[Ditte]
    \role{Ninja}[Søren]
\end{roles}


\begin{props}
    \prop{Busk}[Person, der skaffer]Busk
    \prop{skovl}[Person, der skaffer]Skovl
    \prop{Bog}[Person, der skaffer]Bog
    \prop{Rosenkrans}[Person, der skaffer]Rosenkrans (katolsk)
    \prop{Tonsur-paryk}[Person, der skaffer]Tonsur-paryk
    \prop{Røgelseskar}[Person, der skaffer]Røgelseskar
    \prop{Pistol}[Person, der skaffer]Pistol
\end{props}

\textbf{Bemærkninger til Teknikken:}
Grønt/gult varmt lys
Pistolskudslyd

\textbf{Kommentarer/kontekst fra forfatterne:}
En dokumentar film om den truede art Humanisten efter regeringens uddannelsesudspil. Dog må vi huske at humanister naturligvis er en invasiv og forhistorisk art, og derfor skal udryddes.


\begin{sketch}
\scene{På scenen står DA og er klar til at gå rundt i de naturskønne område og give ud af sin enorme viden om habitatet og dets beboere. Scenen har eventuelt nogle træer (ninjaer?).}
\scene{lys op}
\act{DA} Går langsomt rundt på scenen hele tiden mens han taler og sørger for at gøre plads til de andre når de skal præsenteres.

\says{SP} Det næste program, på programmet, i vores program, er Naturen i Byen med David Attenborough
\says{DA} Godaften allesammen og velkommen til Naturen i Byen. Vi er idag taget til Amager Fælled for at se på en art, der efter flere af regeringens udspil er gået kraftigt tilbage. 
Vi tænker naturligvis på humanisten.
Man har gjort et forsøg for at sikre artens overlevelse ved at konstruere et trygt habitat for humanisterne her tæt på Amager Fælled, også kendt som Søndre Campus. 
Hvis vi prøver at gå herhen kan det være at vi vil få øje på en forbipasserende humanist.
\act{DA} Stiller sig bag ved en busk og kigger efter humanister
\act{H1} Kommer ind på scenen med skovl over skulderen iklædt arkæolog outfit. Går hen til lemmen og begynder at “grave” og hiver kniv, vikingehat og kranie op af lemmen.
\says{DA} Der var vi sørme heldige. Det ligner minsandten en arkæolog. Hvis jeg ikke tager fejl, så er det sågar en klassisk arkæolog. De er sjældnere at se i disse egne end forhistoriske arkæologer, så det her er meget fascinerende. Prøv at se hvordan den prøver at gøre sig berettiget for sit eksistensgrundlag. Er det ik sødt. 
\act{H1} Giver stolt DA kranie
\says{DA} Men det er jo ikke til nogen nytte. Så ned i hullet med den. 
\act{DA} Skubber H1 ned i hullet 
\says{DA} Ja! Dem der graver i andres grav, falder selv i, eller hvordan man nu siger. Desuden er der altid et dilemma om hvor lang tid nogen skal være død før det er arkæologi og ikke gravrøveri eller nekrophili. 
Humanisten vil mere end ofte samlet set ende ud i at nasse på systemet, og derved nedsætte fitness for alle rigtige videnskabsfolk. Man kunne godt kalde det et parasitisk symbiotisk forhold.
Århv! Jeg tror faktisk der kommer en til!
\act{DA} Skynder sig om bag busken igen
\act{H2} Kommer ind på scenen med en bog i hånden (litteraturhistoriker). Får øje på kranie tager det op og diskuterer en eller anden forfatter/bog til det (evt. uden mic).
\says{H2} (uden mic) Hvad laver du dernede? Har du måske læst Gøthes Den Unge Werthers Lidelser? Altså du burde virkelig overveje hvordan du kan implementere Kafkas Forvandlingen i dit verdenssyn.
\says{DA} Her har vi tydeligvis en Litteraturhistoriker. Det kan man se på hvordan den løfter venstre fod når den går. Her er vi samtidig vidne til et spændende fænomen, når de forskellige populationer mødes og interagerer. Dette er et lærebogseksempel et filosofikum. Det er der dog ikke brug for i den moderne verden, så ned i hullet med den.
\act{DA} Skubber H2 i hullet
\says{DA} En af de lyse sider på at se denne art uddø er naturligvis at forskningspengene omsider vil blive frigjort til de ordentlige videnskaber i stedet for det her tant og fjas.
Og sådan er det! Wauw! Prøv lige en gang at se det eksemplar der kommer der!
\act{DA} Gemmer sig bag busken igen
\act{H3} Kommer ind med bibel, tonsur, røgelseskar og andre religiøse artefakter. Messer evt noget volapyk-latin
\says{AV}\scene Torden lyd
\says{DA} Jeg troede ikke at jeg ville komme til at se en vaskeægte teolog længere. Populationen i indre by blev jo for nyligt udryddet, men det ser ud til at der er en nyindvandret population her ved Søndre Campus. Men der er trods alt en grund til at populationen i indre by blev udryddet så…
\act{DA} Stikker H3 med kniv. Skyder H3 (pistollyd)
\act{H3} Falder på jorden og vrøvler videre
\says{DA} shhh \act{DA} Skyder H3 (pistollyd). 
\says{DA} Men som i kan se, så er humanisterne i drastisk tilbagegang. Men selvom den er udrydelsestruet er den ikke nødvendigvis særligt bevaringsværdig. Men det bliver spændende og meget underholdende at følge udviklingen i populationerne her på Søndre Campus i fremtiden. Mit navn er David Attenborough og tusind tak for i aften.  
\says{SP} Følg med næste gang når vi tager til Botanisk Have og udfører populationskontrol på DJØF’erne fra CSS. Har du problemer med store populationer af humanister, så ring på: NNNNNNNN. Så kigger DAvid Attenborough for og han går ikke før de ikke virker. 
\says{AV}\scene nr. på storskærm eventuelt med reklame for DA skadedyrsservice
\scene{lys ned}
\end{sketch}
\end{document}

